\section{附録A}

\subsection{sup の修飾品詞による意味の変更}

一部の sup はそれが修飾する語の品詞(動詞か名詞)によって意味を変える。これを \textbf{sup のポリモーフィズム} と呼ぶ。具体的には sup \emph{-re}\footnote{「~の」} が該当する。

\subsubsection{-re のポリモーフィズム}

便宜的に文を「A-re B」とする。

\begin{itemize}
    \item A=名詞・B=形容詞 \\ → B は形容詞のままである。
    \item A=動詞・B=形容詞 \\ → B が副詞化する。
\end{itemize}

\subsection{外来語の表記・読み}

外来語(例:英語・日本語)の語を r3 で表記するときは原則として外来語のアルファベット転記をそのまま表記する。r3 にない文字(CやLなど)もそのまま写す。

読みは r3 の発音規則にそってできるだけ似せて表現する。テキスト上で記述するときは原表記と共に読みも記述するのが望ましい。

\subsection{日本語への転記}

r3 の発音規則は概ね日本語と共通しているので、発音する通りに日本語(仮名など)に転記すればよい。