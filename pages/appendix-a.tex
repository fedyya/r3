\chapter{文法の補足}

\section{mod・sup の語順}

原則 mod・sup はどのように並び替えても良い。例外は以下の通りである。数字が若いほど優先度が高い。この例外事項は義務的ではない。

\begin{enumerate}
    \item 疑問詞の絡む sup の項 \\ → 常に他の sup の項より後に付加する。
    \item sup \emph{sy-} \\ → 他の sup より先に付加する。
    \item mod \emph{ay-} \\ → 他の mod より先に付加する。
    \item mod \emph{ny-} \\ → 他の mod より先に付加する。
\end{enumerate}

\section{外来語の表記・読み}

外来語(例:英語・日本語)の語を r3 で表記するときは原則として外来語のアルファベット転記をそのまま表記する。r3 にない文字(CやLなど)もそのまま写す。

読みは r3 の発音規則にそってできるだけ似せて表現する。テキスト上で記述するときは原表記と共に読みも記述するのが望ましい。

\section{日本語への転記}

r3 の発音規則は概ね日本語と共通しているので、発音する通りに日本語(仮名など)に転記すればよい。

\section{約物の表記}

\cref{table:list-of-punctuation}が r3 の記述上の役物として用いられる。

\deftablemargin
\begin{table}[H]
    \centering
    \caption{raipara-re ry の約物一覧}
    \label{table:list-of-punctuation}
    \begin{tabular}{ccl}
        \toprule
        役物 & 範囲 & 意味 \\
        \midrule
        .  & 文の終わり & 文の終了記号。 \\
        ,  & 文の区切り & 文を列挙する時や副文の結合記号。 \\
        -  & modの直後・supの直前 & mod・sup を語に結合するための記号。 \\
        !  & 文の終わり & 文の強調を明示的に示す記号。 \\
        ?  & 文の終わり & 疑問文の終わりを明示的に示す記号。 \\
        «» & 文の前後 & 会話・文章を引用・強調するための記号。 \\
        \bottomrule
    \end{tabular}
\end{table}

\section{複合語の形成}