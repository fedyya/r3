\chapter{文法の補足}

\section{mod・sup の語順}

原則 mod・sup はどのように並び替えても良い。例外は以下の通りである。数字が若いほど優先度が高い。この例外事項は義務的ではない。

\begin{multicols}{2}
    \subsection{mod の語順}

    \begin{enumerate}
        \item \emph{ike-}・\emph{yke-}・\emph{san-}・\emph{tan-} \\ → 他の mod より\textbf{先}に置く。
        \item \emph{ay-} \\ → 他の mod より\textbf{先}に置く。
        \item \emph{ny-} \\ → 他の mod より\textbf{先}に置く。
        \item \emph{ho-}・\emph{ro-} \\ → 他の mod の\textbf{後}に置く。
        \item \emph{kyi-} \\ → 他の mod の\textbf{後}に置く。
    \end{enumerate}

    \columnbreak

    \subsection{sup の語順}

    \begin{enumerate}
        \item 疑問詞の絡む sup の項 \\ → 常に他の sup の項の\textbf{後}に置く。
        \item \emph{-e}・\emph{-i} \\ → 他の sup より\textbf{先}に置く。
        \item \emph{sy-} \\ → 他の sup より\textbf{先}に置く。
        \item \emph{hy-} \\ → \emph{-sy} の\textbf{直後}に置く。
        \item \emph{zy-} \\ → 他の sup より\textbf{先}に置く。
    \end{enumerate}
\end{multicols}

\section{約物の表記}

\cref{table:list-of-punctuation}が r3 の記述上の役物として用いられる。

\deftablemargin
\begin{table}[H]
    \centering
    \caption{raipara-re ry の約物一覧}
    \label{table:list-of-punctuation}
    \begin{tabular}{ccl}
        \toprule
        役物 & 範囲 & 意味 \\
        \midrule
        .  & 文の終わり & 文の終了記号。 \\
        ,  & 文の区切り & 文を列挙する時や副文の結合記号。 \\
        -  & modの直後・supの直前 & mod・sup を語に結合するための記号。 \\
        !  & 文の終わり & 文の強調を明示的に示す記号。 \\
        ?  & 文の終わり & 疑問文の終わりを明示的に示す記号。 \\
        «» & 文の前後  & 会話・文章を引用・強調するための記号。 \\
        '  & 適当な箇所 & 連続する母音などを分別するための記号。 \\
        \bottomrule
    \end{tabular}
\end{table}

\section{外来語の表記・読み}

\begin{wraptable}[9]{r}{50mm}
    \centering
    \caption{表記の置換ルール}
    \label{table:r3-rewrite}

    \begin{tabular}{cc}
        \toprule
            置き換え元 & 置き換え先 \\
        \midrule
            b & v \\
            c & k \\
            j & zh \\
            l & r \\
            y & i \\
        \bottomrule
    \end{tabular} \\
    注:英 → r3
\end{wraptable}

外来語(例:英語・日本語)の語を r3 で表記するときは原則として外来語のアルファベット転記をそのまま表記する。
r3 にない文字(CやLなど)は原則そのまま写す。

読みは r3 の発音規則にそってできるだけ似せて表現する。テキスト上で記述するときは原表記と共に読みも記述するのが望ましい。

借用語など外来語を r3 風に直すときは \cref{table:r3-rewrite} のように変換する。

\section{日本語への転記}

r3 の発音規則は概ね日本語と共通しているので、発音する通りに日本語(仮名など)に転記すればよい。

\begin{itembox}[l]{r3 → 日本語への転記例}
    \begin{pindent}
        \noindent
        sy-hy raipara-re ry kene. \\
        「すふ らいぱら・れ る けーね」
    \end{pindent}
\end{itembox}

\section{各種の代替的記述}

母音 \emph{y} は他言語の話者の便宜のために \emph{u} と書いてもよい。ただし \emph{y} と \emph{u} は文章の中に混在させないのが望ましい。

mod・sup の付加に使う記号 \emph{-} は省略してもよい。そのときは文章の一部のみを省略するのではなくすべての使用箇所を省略する。

\section{複合語の形成}