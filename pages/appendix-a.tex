\chapter{附録A}

\section{mod・sup の語順}

原則 mod・sup はどのように並び替えても良い。例外は以下の通りである。数字が若いほど優先度が高い。この例外事項は義務的ではない。

\begin{enumerate}
    \item 疑問詞の絡む sup の項 \\ → 常に他の sup の項より後に付加する。
    \item sup \emph{sy-} \\ → 他の sup より先に付加する。
    \item mod \emph{ay-} \\ → 他の mod より先に付加する。
    \item mod \emph{ny-} \\ → 他の mod より先に付加する。
\end{enumerate}

\section{外来語の表記・読み}

外来語(例:英語・日本語)の語を r3 で表記するときは原則として外来語のアルファベット転記をそのまま表記する。r3 にない文字(CやLなど)もそのまま写す。

読みは r3 の発音規則にそってできるだけ似せて表現する。テキスト上で記述するときは原表記と共に読みも記述するのが望ましい。

\section{日本語への転記}

r3 の発音規則は概ね日本語と共通しているので、発音する通りに日本語(仮名など)に転記すればよい。