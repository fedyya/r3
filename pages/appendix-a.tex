\chapter{文法の補足}

\section{状態詞・供与詞の語順}

状態詞・供与詞はどのように並び替えても良い。しかし、推奨される順序は以下の通りである。数字が若いほど優先度が高い。

\begin{multicols}{2}
    \subsection{状態詞の語順}

    \begin{enumerate}
        \item 接続 \emph{sed}・\emph{ned}・\emph{ad}
        \item 疑問文宣言 \emph{aus}
        \item 数 \emph{nos}・\emph{dos}・\emph{nas}
        \item 性 \emph{ar}・\emph{er}・\emph{nir}
        \item 態・相 \emph{rit}・\emph{sit}・\emph{niz}
        \item 法 \emph{fis}・\emph{sis}・\emph{zis}
        \item 時制 \emph{ak}・\emph{ek}
        \item 否定 \emph{es}・\emph{ves}
    \end{enumerate}

    \columnbreak

    \subsection{供与詞の語順}

    \begin{enumerate}
        \item 主格 \emph{k}
        \item 対格 \emph{s}
        \item 奪格 \emph{ruz}
        \item 与格 \emph{zek}
        \item 随伴 \emph{vei}・\emph{nei}
        \item ここに示されない供与詞
        \item 疑問詞の項
        \item 接続 \emph{ren}・\emph{den}
    \end{enumerate}
\end{multicols}

\section{約物の表記}

\cref{table:list-of-punctuation}が r3 の記述上の役物として用いられる。

\deftablemargin
\begin{table}[H]
    \centering
    \caption{raipara-re ry の約物一覧}
    \label{table:list-of-punctuation}
    \begin{tabular}{ccl}
        \toprule
        役物 & 範囲 & 意味 \\
        \midrule
        .  & 文の終わり & 文の終了記号。 \\
        ,  & 文の区切り & 文を列挙する時や副文の結合記号。 \\
        !  & 文の終わり & 文の強調を明示的に示す記号。 \\
        ?  & 文の終わり & 疑問文の終わりを明示的に示す記号。 \\
        «» & 文の前後  & 会話・文章を引用・強調するための記号。 \\
        '  & 適当な箇所 & 発音の区切りを定める記号。 \\
        \bottomrule
    \end{tabular}
\end{table}

\section{外来語の表記・読み}

外来語(例:英語・日本語)の語を表記するときは原則として外来語のアルファベット転記をそのまま表記する。
言語にない文字(CやLなど)もそのまま写す。

読みは言語の発音規則にそってできるだけ似せて表現する。テキスト上で記述するときは原表記と共に読みも記述するのが望ましい。

\section{日本語への転記}

発音規則は概ね日本語と共通しているので、発音する通りに日本語(仮名など)に転記すればよい。

\begin{itembox}[l]{r3 → 日本語への転記例}
    \begin{pindent}
        \noindent
        Aikas \emph{Paranik Erka} nirkaz foska. \\
        「あいかす ぱらにーく えるか にるかず ふぉすか」 \\
        「パラニーク・エルカは光であり力である」
    \end{pindent}
\end{itembox}

\section{複合語の形成}