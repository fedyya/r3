\section{形態}

\subsection{Modifier}

\textbf{Modifier} は動詞に法・相・態などを追加するものである。
Modifier は語の前につく(例. \emph{njy-}kyvenai「知らない」)。


\begin{table}[h]
    \centering
    \caption{主たるModifierの表}
    \begin{tabular}{ccll}
        \hline
        語 & 意味 & 用例 & 備考 \\
        \hline \hline
        seja- & ~しなさい(強) & \emph{seja-}zjarai! & 上司が命令するイメージ \\
        nara- & ~しなさい(弱) & \emph{nara-}pjanai! & 友人を誘うイメージ\\
        mja-  & ~ならば  & \emph{mja-}dai-sy re, ...& \\
        waie- & ~している & \emph{waie-}tarjai. & \\
        roi-  & 過去、~ & \emph{roi-}kerai. & waie- と併用で「~していた」\\
        hoi-  & 将来、~ & \emph{hoi-}derai. & \\
        \hline
    \end{tabular}
\end{table}

\subsection{Supplier}

\textbf{Supplier} は動詞・名詞に続く語の定義を表すものである。
Supplier は語の後につく(例:rjai\emph{-sy} re.)。

以下の表に出るXはそれぞれの Supplier が取る語を指す。

\begin{table}[h]
    \centering
    \caption{主たるSupplierの表}
    \begin{tabular}{ccll}
        \hline
        語 & 意味 & 用例 & 備考 \\
        \hline \hline
        -sy  & 主語定義 & monai\emph{-sy} ry. & X がいわゆる主語となる \\
        -sei & ~を & hanai\emph{-sei} ro.           & \\
        -rei & ~の & verai-sei vera\emph{-rei} hyraiza.   & X は名詞句でも動詞句でもよい \\
        -ray & ~から & kewai\emph{-ray} ahyra.            & \\
        -zay & ~で & wawai\emph{-zay} iy.           & X は場所 \\
        -tay & ~へ & zarai\emph{-tay} anjea.              & X は場所 \\
        \hline
    \end{tabular}
\end{table}

\subsection{動詞}

動詞は語幹に接尾辞 \textbf{-ai} がついた形で表される。動詞は modifier と supplier を持つ。

\subsection{名詞}

名詞は語幹に接尾辞 \textbf{-a} がついた形で表される。名詞は supplier を持つ。

\subsection{形容詞}

形容詞は語幹に接尾辞 \textbf{-e} がついた形で表される。形容詞は適切な supplier によって付加される。