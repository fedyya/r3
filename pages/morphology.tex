\chapter{形態論}

\section{状態詞}

\textbf{状態詞}(略記:mod\footnote{英訳 \emph{Modifier} の略。})は動詞に法・相・態などを追加するものである。
状態詞は語の前につく(例. \emph{ny-}kyvenai「知らない」)。状態詞には特定の接尾辞がつかない。

\cref{table:common-mods}はよく使われる状態詞の一覧である。

\begin{table}[H]
    \centering
    \caption{主な状態詞の表}
    \label{table:common-mods}
    \begin{tabular}{ccll}
        \toprule
        語 & 意味 & 用例 & 備考 \\
        \midrule
        di-  & ~しなさい(強) & \emph{di-}zhavai! & 上司が命令するイメージ \\
        ri-  & ~しなさい(弱) & \emph{ri-}panhai! & 友人を誘うイメージ\\
        ni-  & ~ならば   & \emph{ni-}dai-sy re, ...& \\
        kyi- & ~している & \emph{kyi-}monai. & \\
        ro-  & 過去、~   & \emph{ro-}oreai. & kyi- と併用で「~していた」\\
        ho-  & 将来、~   & \emph{ho-}derai. & \\
        \bottomrule
    \end{tabular}
\end{table}

\begin{itembox}[l]{状態詞をひたすら並べてみた例}
    \begin{pindent}
        tan-ny-esi-kyi-ho-nean-zeai. \\
        tan\supsub{mod.}{そして~}-
        ny\supsub{mod.}{~しない}-
        esi\supsub{mod.}{~される}-
        kyi\supsub{mod.}{~している}-
        ho\supsub{mod.}{~だった}-
        nean\supsub{mod.}{~としても}-
        zeai\supsub{v.}{愛する} \\
        「そして愛されていなかったとしても」
    \end{pindent}
\end{itembox}

\section{供与詞}

\textbf{供与詞}(略記:sup\footnote{英訳 \emph{Supplier} の略。})は動詞・名詞に続く語の関係を示すものである(詳細は統語論で扱う)。供与詞は語の後につく(例:rhai\emph{-se} re.)。供与詞には特定の接尾辞がつかない。

供与詞が動詞・名詞に付くことを「\textbf{修飾(する)}」と呼ぶ。語に供与詞がついていることを「(語が供与詞を)\textbf{持つ}」という。供与詞が取る語(例:vea-re veie ならば供与詞 \emph{-re} に対して \emph{veie})を「\textbf{項}」と呼ぶ。供与詞が項を取り関係を確定させることを「(項を)\textbf{解決する}」と言う。

\cref{table:common-sups}はよく使われる供与詞の一覧である。

\begin{table}[H]
    \centering
    \caption{主な供与詞の表}
    \label{table:common-sups}
    \begin{tabular}{cclll}
        \toprule
        語 & 意味 & 用例 & 意味 \\
        \midrule
        -sy  & ~は & monai\emph{-sy} ry.                & 「私は寝る」\\
        -se  & ~を & heriai\emph{-se} ina.              & 「これを奏でる」\\
        -re  & ~の & parai-se parassa\emph{-re} ferina. & 「本当の答えを言う」\\
        -nie & ~で & rerei\emph{-nie} ina.              & 「ここで遊ぶ」 \\
        \bottomrule
    \end{tabular}
\end{table}

\section{名詞}

名詞は接尾辞 \textbf{-a} がついた形で表される。ただし一部の名詞(専ら代名詞の一部で \emph{ry} や \emph{re} など)は -a で終わらない。名詞は供与詞を持てる。

\subsection{人称代名詞}

\begin{wraptable}[6]{r}{50mm}
    \centering
    \caption{数・文脈で変化する人称代名詞}
    \label{table:list-of-re-rea}
    \begin{tabular}{ccl}
        \toprule
        語 & 意味 & 文脈 \\
        \midrule
        re    & 貴方  & フォーマル \\
        reky  & 貴方達 & 同上 \\
        rea   & 君    & インフォーマル \\
        rena  & 君達  & 同上 \\
        \bottomrule
    \end{tabular}
\end{wraptable}

人称代名詞は一人称、二人称、三人称の3つが存在する。扱い・表記共に名詞に準ずるが、すべての人称代名詞が \emph{-a} で終わるとは限らず特に一人称・二人称に例外が多数存在する。

また、文脈に応じて人称代名詞を使い分けることもできる。フォーマル・インフォーマルな言い方や、文法には本来存在しない数を補うものとして単数・複数それぞれの表現が用意されているものが存在する。

\subsection{所有代名詞}

\section{動詞}

動詞は語幹に接尾辞 \textbf{-ai}、\textbf{-ei}、\textbf{-oi} のいずれかがついた形で表される。動詞は状態詞と供与詞を持てる。

\subsection{繋辞 -hy・存在 -zy}

動詞 \emph{ai} に繋辞の供与詞 \emph{-hy} を修飾させることで、「(Aは)B である」と表現できる。また、存在を示す供与詞 \emph{-zy} を修飾させることで、「A がいる(存在する)」と表現できる。

\begin{itembox}[l]{単純な「AはBである」の例}
    \begin{pindent}
        \noindent
        ai-sy-hy re perie. \\
        ai\supattr{v.}-sy\supsub{sup.}{nom.,\#1}-hy\supsub{sup.}{acc.,\#2}
            re\supsub{pron.}{nom.,\#1} perie\supsub{adj.}{acc.,\#2} \\
        「あなたは可愛い」
    \end{pindent}
\end{itembox}

\begin{itembox}[l]{単純な「Aがある」の例}
    \begin{pindent}
        \noindent
        ai-zy desia. \\
        ai\supattr{v.}-zy\supsub{sup.}{nom.,\#1} desia\supsub{n.}{nom.,\#1}.\\
        「希望がある」
    \end{pindent}
\end{itembox}

また、\cref{section:sentence-without-noun-verb}を利用することで動詞 \emph{ai} を省略することもできる。

\begin{itembox}[l]{動詞 \emph{ai} の省略}
    \begin{pindent}
        \noindent
        sy-hy re perie.「あなたは可愛い」

        \noindent
        zy desia.「希望がある」
    \end{pindent}
\end{itembox}

\subsection{項になることによる動名詞化}

動詞が供与詞 \emph{-se}, \emph{-re} などの項となったとき動詞は動名詞のように働く。
動名詞はもっぱら「~すること」という意味になる。

\begin{itembox}[l]{動名詞と化した動詞}
    \begin{pindent}
        \noindent
        zeai-sy-se ry parai. \\
        zeai\supattr{v.}-sy\supsub{sup.}{nom.,\#1}-se\supsub{sup.}{acc.,\#2}
            ry\supsub{pron.}{nom,\#1} parai\supsub{v.}{acc.,\#2} \\
        「私は話すことが好き」
    \end{pindent}
\end{itembox}

\subsection{動詞の名詞・分詞化}

\begin{wraptable}[7]{r}{85mm}
    \centering
    \caption{動詞の名詞・分詞化}
    \label{table:conjugation-of-verb}
    \begin{tabular}{cccc}
        \toprule
        動詞の接尾辞 & 動詞 & 名詞 & 形容詞・副詞 \\
        \midrule
        -ai & -ai & -a  & -ae \\
        -ei & -ei & -ea & -e  \\
        -oi & -oi & -a  & -e \\
        \bottomrule
    \end{tabular}
\end{wraptable}

すべての動詞はそれぞれ名詞の形を持っている。名詞化した動詞は動名詞とは異なり、意味合いも実際の形態も異なる。いくつかの動詞は分詞(形容詞・副詞として扱われる)の形も持っている。名詞・分詞への変換は動詞の語尾に応じて\cref{table:conjugation-of-verb}のようにする。

供与詞 \emph{-re} などで動詞を他の名詞に修飾したり、動詞自体を名詞として主語に取ったり(項に動詞を取る供与詞 \emph{-sy} 等)するときは適量活用するのが望ましい。動詞を名詞化して供与詞 \emph{-re} などで項に取った場合は名詞の意味で解決され、動詞のままで項に取った場合は「~すること」という意味で解決される。以下は例である。

\begin{itembox}[l]{動詞が形容詞に転用された例}
    \begin{pindent}
        \noindent
        senoa-re danoa. \\
        senoa\supsub{n.}{nom.}-re\supsub{sup.}{gen.,\#1} danoa\supsub{n.}{gen.,\#1}. \\
        「名前の意義」
    \end{pindent}
\end{itembox}

\begin{itembox}[l]{動詞がそのまま \emph{-re} の項になる例}
    \begin{pindent}
        senoa-re danoi. \\
        senoa\supsub{n.}{nom.}-re\supsub{sup.}{gen.,\#1} danoi\supsub{v.}{gen,\#1}. \\
        「名付けることの意味」
        \noindent
    \end{pindent}
\end{itembox}

\section{形容詞}

形容詞は接尾辞 \textbf{-e} がついた形で表される。これには動詞が形容詞化したものも含まれる。形容詞の内、\textbf{名詞を修飾する} 供与詞 \emph{-re} によって付加されたものが正確な意味での形容詞である。一方、動詞を修飾する形容詞は副詞として扱われる。

\section{副詞}

副詞は形容詞が\textbf{動詞を修飾する}供与詞 \emph{-re} によって付加されたものである。即ち、名詞に掛かる \emph{-re} による形容詞は形容詞のままで、動詞に掛かるものは形容詞が副詞として扱われる。

\begin{itembox}[l]{名詞に形容詞が修飾された例}
    \begin{pindent}
        \noindent
        deria-re perie. \vspace{-1mm} \\
        deria\supsub{n.}{nom.} -re\supsub{sup.}{gen,\#1} perie\supsub{adj.}{gen,\#1} \vspace{-1mm} \\
        「可愛い人々」
    \end{pindent}
\end{itembox}

\begin{itembox}[l]{動詞に形容詞が修飾された例}
    \begin{pindent}
        \noindent
        pakorai-re perie. \vspace{-1mm} \\
        pakorai\supattr{v.} -re\supsub{sup.}{gen,\#1} perie\supsub{adv.}{gen,\#1} \vspace{-1mm} \\
        「可愛く笑う」
    \end{pindent}
\end{itembox}

\section{疑問詞}

\begin{wraptable}[6]{r}{40mm}
    \centering
    \label{table:common-interrogatives}
    \caption{主な疑問詞}
    \begin{tabular}{cl}
        \toprule
        疑問詞 & 意味 \\
        \midrule
        zaria   & 誰 \\
        zarea   & 何 \\
        zarara  & 何処 \\
        zavea   & 何時 \\
        zarezea & どのように \\
        \bottomrule
    \end{tabular}
\end{wraptable}

疑問詞は接頭辞 \emph{za-} がついた形で表される。他は名詞に準ずる。

すべての名詞は接頭辞をつけることで疑問詞に転用できる。ただし転用したことによる意味の変遷に一定の規則はない。\cref{table:common-interrogatives}以外の疑問詞は一意な意味を持たないので、意味の確定は話者に託される。

\section{時制}

時制は状態詞で表現する。過去は状態詞 \emph{ro-}、未来は状態詞 \emph{ho-} を付加して表現する。時制の表現は何かを行ったときの時間以外の意味を含まない。いわゆる過去進行形、未来進行形は相の表現と併用して表現できる。

\section{相}

相は状態詞で表現する。

継続相は状態詞 \emph{kyi-} を付加する。

\section{受動態・使役態}

態は状態詞で表現する。

能動態のときは無標である。受動態は状態詞 \emph{esi-} を付加する。使役態は状態詞 \emph{eki-} を付加する。