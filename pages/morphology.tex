\chapter{形態論}

\section{Modifier・Supplier}

\subsection{Modifier}
\textbf{Modifier} (略記:mod\footnote{supを含めてすべて小文字で表記する})は動詞に法・相・態などを追加するものである。mod は語の前につく(例. \emph{ny-}kyvenai「知らない」)。mod には特定の接尾辞がつかない。\cref{table:common-mods}はよく使われる mod の一覧である。

\begin{table}[H]
    \centering
    \caption{主なModifierの表}
    \label{table:common-mods}
    \begin{tabular}{ccll}
        \toprule
        語 & 意味 & 用例 & 備考 \\
        \midrule
        di-  & ~しなさい(強) & \emph{di-}zhavai! & 上司が命令するイメージ \\
        ri-  & ~しなさい(弱) & \emph{ri-}panhai! & 友人を誘うイメージ\\
        ni-  & ~ならば   & \emph{ni-}dai-sy re, ...& \\
        kyi- & ~している & \emph{kyi-}monai. & \\
        ro-  & 過去、~   & \emph{ro-}oreai. & kyi- と併用で「~していた」\\
        ho-  & 将来、~   & \emph{ho-}derai. & \\
        \bottomrule
    \end{tabular}
\end{table}

\subsection{Supplier}

\textbf{Supplier}(略記:sup) は動詞・名詞に続く語の関係を示すものである(詳細は統語論で扱う)。sup は語の後につく(例:rhai\emph{-se} re.)。sup には特定の接尾辞がつかない。

sup が動詞・名詞に付くことを「\textbf{修飾(する)}」と呼ぶ。語に sup がついていることを「(語が sup を)\textbf{持つ}」という。sup が取る語(例:vea-re veie ならば sup \emph{-re} に対して \emph{veie})を「\textbf{項}」と呼ぶ。sup が項を取り関係を確定させることを「(項を)\textbf{解決する}」と言う。

\cref{table:common-sups}はよく使われる sup の一覧である。

\begin{table}[H]
    \centering
    \caption{主なSupplierの表}
    \label{table:common-sups}
    \begin{tabular}{cclll}
        \toprule
        語 & 意味 & 用例 & 意味 \\
        \midrule
        -sy  & ~は & monai\emph{-sy} ry.                & 「私は寝る」\\
        -se  & ~を & heriai\emph{-se} ina.              & 「これを奏でる」\\
        -re  & ~の & parai-se parassa\emph{-re} ferina. & 「本当の答えを言う」\\
        -nie & ~で & rerei\emph{-nie} ina.              & 「ここで遊ぶ」 \\
        \bottomrule
    \end{tabular}
\end{table}

\subsection{mod・sup の語順}

原則 mod・sup はどのように並び替えても良い。例外は以下の通りである。数字が若いほど優先度が高い。この例外事項は義務的ではない。

\begin{enumerate}
    \item 疑問詞の絡む sup の項 \\ → 常に他の sup の項より後に付加する。
    \item sup \emph{sy-} \\ → 他の sup より先に付加する。
    \item mod \emph{ay-} \\ → 他の mod より先に付加する。
    \item mod \emph{ny-} \\ → 他の mod より先に付加する。
\end{enumerate}

\section{動詞}

動詞は語幹に接尾辞 \textbf{-ai}、\textbf{-ei}、\textbf{-oi} のいずれかがついた形で表される。動詞は mod と sup を持つ。

\subsection{動詞の名詞・形容詞・副詞化}

\begin{wraptable}[8]{r}{85mm}
    \centering
    \caption{動詞の名詞・形容詞・副詞への活用}
    \label{table:conjugation-of-verb}
    \begin{tabular}{cccc}
        \toprule
        動詞の接尾辞 & 動詞 & 名詞 & 形容詞・副詞 \\
        \midrule
        -ai & -ai & -a  & -ae \\
        -ei & -ei & -ea & -e  \\
        -oi & -oi & -oa & -oe \\
        \bottomrule
    \end{tabular}
\end{wraptable}

すべての動詞はそれぞれ名詞の形を持っている。いくつかの動詞は形容詞・副詞の形も持っている。活用は動詞の語尾に応じて\cref{table:conjugation-of-verb}のようにする。

sup \emph{-re} などで動詞を他の名詞に修飾したり、動詞自体を名詞として主語に取ったり(項に動詞を取る sup \emph{-sy} 等)するときは適量活用するのが望ましい。動詞を名詞化して sup \emph{-re} などで項に取った場合は名詞の意味で解決され、動詞のままで項に取った場合は「~すること」という意味で解決される。以下は例である。

\begin{multicols}{2}
    \begin{pindent}
        \noindent
        senoa-re danoa. \\
        senoa\supsub{n.}{nom.}-re\supsub{sup.}{gen.,\#1} danoa\supsub{n.}{gen.,\#1}. \\
        「名前の意義」
    \end{pindent}

    \columnbreak

    \begin{pindent}
        senoa-re danoi. \\
        senoa\supsub{n.}{nom.}-re\supsub{sup.}{gen.,\#1} danoi\supsub{v.}{gen,\#1}. \\
        「名付けることの意味」
        \noindent
    \end{pindent}
\end{multicols}

\section{名詞}

名詞は接尾辞 \textbf{-a} がついた形で表される。名詞は sup を持つ。ただし一部の名詞(専ら代名詞の一部で \emph{ry} や \emph{re} など)は -a で終わらない。

副詞は形容詞が\textbf{動詞に掛かる} sup \emph{-re} によって付加されたものである。即ち、名詞に掛かる \emph{-re} による形容詞は形容詞のままで、動詞に掛かるものは形容詞が副詞として扱われる。

\begin{multicols}{2}
    \begin{pindent}
        \noindent
        deria-re perie. \vspace{-1mm} \\
        deria\supsub{n.}{nom.} -re\supsub{sup.}{gen,\#1} perie\supsub{adj.}{gen,\#1} \vspace{-1mm} \\
        「可愛い人々」
    \end{pindent}

    \columnbreak

    \begin{pindent}
        \noindent
        pakorai-re perie. \vspace{-1mm} \\
        pakorai\supattr{v.} -re\supsub{sup.}{gen,\#1} perie\supsub{adv.}{gen,\#1} \vspace{-1mm} \\
        「可愛く笑う」
    \end{pindent}
\end{multicols}

\section{疑問詞}

\section{時制・相・法・態}

\subsection{時制}

時制は mod で表現する。過去は mod \emph{ro-} で、未来は mod \emph{ho-} を付加して表現する。

\subsection{継続相}

相は mod で表現する。どの相も特定の時制を表現することはない。時制はそれ専用の mod を別途付加することで表現する。

継続相は mod \emph{kyi-} を付加する。

\subsection{受動態・使役態}

態は mod で表現する。能動態のときは無標である。受動態は mod \emph{esi-} を付加する。使役態は mod \emph{eki-} を付加する。