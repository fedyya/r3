\chapter{形態論}

\section{状態詞}

\textbf{状態詞}(略記:mod\footnote{英訳 \emph{Modifier} の略。})は語の状態を示すものであり、必ず子音で終わる形で表記される。状態詞は語の前につく(例. \emph{es}nisri「会わない」)。

\cref{table:common-mods}はよく使われる状態詞の一覧である。

\begin{table}[H]
    \centering
    \caption{主な状態詞の表}
    \label{table:common-mods}
    \begin{tabular}{ccll}
        \toprule
        語 & 意味 & 用例 & 意味 \\
        \midrule
        ek  & 過去形 & \emph{ek}pasri. & 「奏でた」\\
        ak  & 未来形 & \emph{ak}senzi. & 「変わるだろう」 \\
        niz & ~している & \emph{niz}mesri. & 「待っている」 \\
        zis & ~しなさい & \emph{zis}nekri! & 「離れなさい!」\\
        sis & ~しよう & \emph{sis}vanti! & 「行こう!」\\
        \bottomrule
    \end{tabular}
\end{table}

\subsection{数・性・相・態・時制の表現}

% TODO: TBA

\section{供与詞}

\textbf{供与詞}(略記:sup\footnote{英訳 \emph{Supplier} の略。})語と語の関係を示すものである(詳細は統語論で扱う)。供与詞は語の後につく(例:sari\emph{k} ne.「貴方を思う」)。供与詞には特定の接尾辞がつかない。

複数の供与詞を一つの語に修飾することは可能でありごく一般的に行われる(例・「……『を』~~『に』」→ -k-zek)。
適当な順番で修飾していけばよいが、このとき供与詞と供与詞の間に \emph{-a-} が挿入される(例:-kazek)。

\cref{table:common-sups}はよく使われる供与詞の一覧である。

\begin{table}[H]
    \centering
    \caption{主な供与詞の表}
    \label{table:common-sups}
    \begin{tabular}{cclll}
        \toprule
        語 & 意味 & 用例 & 意味 \\
        \midrule
        k   & ~は & Nasi\emph{k} ni.     & 「私は探す」\\
        s   & ~を & Pasri\emph{s} astera. & 「夢を奏でる」\\
        nik & ~の & Faira\emph{nik} vismike & 「赤い世界」\\
        \bottomrule
    \end{tabular}
\end{table}

\subsection{供与詞で使われる用語}

供与詞周辺の処理は複雑であるので、以下のように用語を定める。
なお、例文として \emph{Sarik ne.}(意味:「私は思う」を使用する。

\begin{itemize}
    \item 供与詞が語に付くことを「\textbf{修飾(する)}」という。 \\ 例文の供与詞 \emph{k} は動詞 \emph{sari} を修飾している。
    \item 供与詞から見て修飾された語を「\textbf{接受語}」という。\\ 例文の動詞 \emph{sari} は 供与詞 \emph{k} の接受語である。
    \item 語に供与詞がついていることを「(語が供与詞を)\textbf{持つ}」という。 \\ 例文の動詞 \emph{sari} は供与詞 \emph{k} を持っている。
    \item 供与詞が取る語を「\textbf{項}」という。 \\ 例文の代名詞 \emph{ne} は供与詞 \emph{k} の項である。
    \item 供与詞を適切に処理して意味を確定させることを「\textbf{解決する}」という。 \\例文はもっとも左側かつ唯一の供与詞である \emph{k} が名詞 \emph{ne} を項に取ることで解決する。
\end{itemize}

\subsection{格の表現}

% TODO: TBA

\subsection{接続詞}

% TODO: TBA

\section{名詞}

名詞は接尾辞 \textbf{-a} がついた形で表される。ただし一部の代名詞(\emph{ni} や \emph{ne} など)は -a で終わらない。

\subsection{人称代名詞}

\begin{wraptable}[8]{r}{50mm}
    \centering
    \caption{主要な代名詞の一覧}
    \label{table:list-of-pron}
    \begin{tabular}{cc}
        \toprule
        語 & 意味 \\
        \midrule
        ni(s) & 私 \\
        ne(s) & 君 \\
        na(s) & これ \\
        sa(s) & それ \\
        ma(s) & あれ \\
        \bottomrule
    \end{tabular}
\end{wraptable}

人称代名詞は一人称、二人称、三人称の3つが存在する。

数・性は一般的に状態詞で表現される一方で、\cref{table:list-of-pron}のような代名詞には語形変化による数の表現が存在する。
例えば「私」の場合、\emph{ni} は単数であるが \emph{nis} になると複数になる。
\subsection{所有代名詞}

\section{動詞}

動詞は語幹に接尾辞 \textbf{-i} がついた形で表される。

\subsection{動詞の転用}

すべての動詞は必ず\textbf{名詞の形を持つ}。また一部の動詞は形容詞・副詞の形も持つ。
動詞が名詞や形容詞、副詞になることを\textbf{転用}という。

名詞に転用された動詞は動名詞とは異なり、意味合いも記述上の形態も異なる。
動詞を名詞に転用して、供与詞 \emph{nik} などで項に取った場合は名詞の意味で解決される。
転用せずそのまま項に取った場合は動名詞(Vすること)または単純な修飾(Vする何々)として意味で解決される。

\begin{itembox}[l]{動詞が形容詞に転用された例}
    \begin{pindent}
        \noindent
        nirkanik \textcolor{magenta}{nekr\textbf{a}}. \\
        「\textcolor{magenta}{遠い}光」
    \end{pindent}
\end{itembox}

\begin{itembox}[l]{動詞がそのまま \emph{nik} の項になる例}
    \begin{pindent}
        nirkanik \textcolor{magenta}{nekr\textbf{i}}. \\
        「\textcolor{magenta}{遠のく}光」
        \noindent
    \end{pindent}
\end{itembox}

\section{形容詞}

形容詞は接尾辞 \textbf{-e} がついた形で表される。これは動詞が転用によって形容詞化したものも含む。このうち \textbf{名詞}を修飾する供与詞 \emph{nik} の項となっているものが正確な意味での形容詞である。

なお、動詞を修飾する形容詞は副詞として扱われる。

\section{副詞}

副詞は形容詞が\textbf{動詞}を修飾する供与詞 \emph{nik} によって付加されたものである。

形容詞のうち動詞に掛かるものが副詞となり、名詞に掛かるものは形容詞のままとなる。

\begin{itembox}[l]{名詞に形容詞が修飾された例}
    \begin{pindent}
        \noindent
        \textcolor{magenta}{pasra}nik vanse. \\
        「悲しい\textcolor{magenta}{演奏}」
    \end{pindent}
\end{itembox}

\begin{itembox}[l]{動詞に形容詞が修飾された例}
    \begin{pindent}
        \noindent
        \textcolor{magenta}{pasri}nik vanse. \\
        「悲しく\textcolor{magenta}{奏でる}」
    \end{pindent}
\end{itembox}

\section{疑問詞}

% TODO: TBA
