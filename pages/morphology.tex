\section{形態論}

\subsection{Modifier・Supplier}

\subsubsection{Modifier}
\textbf{Modifier} (略記:mod\footnote{sup を含めてすべて小文字で表記する})は動詞に法・相・態などを追加するものである。
Modifier は語の前につく(例. \emph{njy-}kyvenai「知らない」)。

\begin{table}[h]
    \centering
    \caption{主たるModifierの表}
    \begin{tabular}{ccll}
        \hline
        語 & 意味 & 用例 & 備考 \\
        \hline \hline
        dea-  & ~しなさい(強) & \emph{dea-}zjavai! & 上司が命令するイメージ \\
        rea-  & ~しなさい(弱) & \emph{rea-}panjai! & 友人を誘うイメージ\\
        fjoa- & ~ならば  & \emph{fjoa-}dai-sy re, ...& \\
        kyi-  & ~している & \emph{kyi-}monai. & \\
        ro-   & 過去、~ & \emph{ro-}oreai. & kyi- と併用で「~していた」\\
        ho-   & 将来、~ & \emph{ho-}derai. & \\
        \hline
    \end{tabular}
\end{table}

\subsubsection{Supplier}

\textbf{Supplier}(略記:sup) は動詞・名詞に続く語の定義を表すものである。
Supplier は語の後につく(例:rjai\emph{-sei} re.)。

以下の表に出るXはそれぞれの Supplier が取る語を指す。

\begin{table}[h]
    \centering
    \caption{主たるSupplierの表}
    \begin{tabular}{ccll}
        \hline
        語 & 意味 & 用例 & 備考 \\
        \hline \hline
        -sy   & 主語定義 & monai\emph{-sy} ry. & X がいわゆる主語となる \\
        -sei  & ~を & hanai\emph{-sei} ro.           & \\
        -rei  & ~の & verai-sei vera\emph{-rei} hyraiza.   & X は名詞句でも動詞句でもよい \\
        -rae & ~から & kewai\emph{-ray} ahyra.            & \\
        -mae & ~で & wawai\emph{-zay} iy.           & X は場所 \\
        -tae & ~へ & zarai\emph{-tay} anjea.              & X は場所 \\
        \hline
    \end{tabular}
\end{table}

\subsubsection{mod・sup の類}
mod・sup はそれぞれの関心に応じて表記が決定される。
これを\textbf{類}といい、例えば hEI類(対象を示す sup の類)などという。

類に示される記号のうち小文字の r\footnote{\textbf{r}ehona 母音} と h\footnote{\textbf{h}ehona 子音} はそれぞれ任意の母音と子音を指し、大文字はその文字自体を指す。
hEI の場合は rei, sei, dei などが当てはまることになる。

\subsection{動詞}

動詞は語幹に接尾辞 \textbf{-ai} がついた形で表される。動詞は mod と sup を持つ。

\subsubsection{動詞の態}

動詞のうち能動態のときは無標である。
受動態のときは mod \emph{zemi-} を動詞に付加する。
使役態のときは mod \emph{nemi-} を動詞に付加する。

\begin{itemize}
    \item zemi-kyvenai-sy-nei minura-rei ry deria.(私の秘密が人々に知られる)
    \item nemi-kaahai-sy-sei re niha.(貴方は鳥を飛ばさせる)
\end{itemize}

\subsubsection{繋辞 hy-}

\subsection{名詞}

名詞は語幹に接尾辞 \textbf{-a} がついた形で表される。名詞は sup を持つ。
代名詞・疑問詞もこれに従う。

\subsection{形容詞}

形容詞は語幹に接尾辞 \textbf{-e} がついた形で表される。形容詞は適切な sup によって付加される。