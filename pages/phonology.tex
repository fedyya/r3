\section{音韻・音声}

\subsection{母音}

\begin{table}[h]
    \centering
    \caption{母音表}
    \begin{tabular}{c||cc}
        \hline
           & 前舌 & 後舌 \\
        \hline \hline
        狭 & \ipa{i}{i}                & \ipa{y}{u} \\
        半狭 & \ipa{e}{e} \ipa{ee}{e:} & \ipa{o}{o} \ipa{oo}{o:} \\
        広 & \ipa{a}{a} \ipa{aa}{a:}   & \\
        \hline
    \end{tabular}
\end{table}

\subsection{子音}

\begin{table}[h]
    \centering
    \caption{子音表(-j 以外)}
    \begin{tabular}{c||cccc}
        \hline
             & 両唇音 & 舌頂音 & 舌背音 & 声門音 \\
        \hline \hline
        破裂音   & \ipa{p}{p} \ipa{v}{b} & \ipa{t}{t} \ipa{d}{d} & \ipa{k}{k} \ipa{g}{g} & \\
        鼻音     & \ipa{m}{m}            & \ipa{n}{n}            &                       & \\
        ふるえ音 &                       & \ipa{r}{r}            &                       & \\
        摩擦音   &                       & \ipa{s}{s} \ipa{z}{z} &                       & \ipa{h}{h} \\
        \hline
    \end{tabular}
\end{table}

\begin{table}[h]
    \centering
    \caption{子音表(-h-)}
    \begin{tabular}{ccccc}
        \hline
        \ipa{th}{\t{tS}}     & \ipa{zh}{\t{dZ}}      & \ipa{rh}{r\super{j}} & \ipa{sh}{\c{c}}       & \ipa{f}{F} \\
        \ipa{ph}{p\super{j}} & \ipa{vh}{b\super{j}}  & \ipa{kh}{k\super{j}} & \ipa{gh}{g\super{j}}  &
        \ipa{mh}{m\super{j}} \\
        \ipa{nh}{\textltailn} &                      &                       & \\
        \hline
    \end{tabular}
\end{table}

\subsection{促音・撥音}

子音を重ねると促音になる。例えば rakka のとき発音は \textipa{/ra.k:a/} になる。
また、撥音に \ipa{n}{\textsyllabic{n}} が存在する。

\subsection{子音そのものの発音}

例えば子音 K や子音 M そのものを発音する(英語で言う「エー、ビー、シー……」)ときは、
\begin{itemize}
    \item 大文字の場合 \emph{-aa}
    \item 小文字の場合は \emph{-ea}
\end{itemize}
を子音につけて発音する。