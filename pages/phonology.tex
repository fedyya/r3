\chapter{音韻・音声}

\section{パラニーク・エルカの書記と発声}

パラニーク・エルカの書記はアルファベット(\emph{Arfabeta})による。このテキストでの書記はアルファベットのうちラテン文字によるが、音韻(音声)と表記が一対一で対応が取れるものであれば他の文字(例:キリル文字)によってもよい。

音韻は\cref{table:vowels}・\cref{table:consonants}・\cref{table:consonants-h}のとおりである。\cref{table:consonants-h}は拗音(「ラ\textbf{ッ}パ」「リ\textbf{ッ}トル」など)である。拗音は必ず \emph{-h-} という接中辞がある。

\section{母音・子音}

\begin{table}[H]
    \centering
    \caption{パラニーク・エルカでの母音}
    \label{table:vowels}
    \begin{tabular}{c|cc}
        \toprule
           & 前舌 & 後舌 \\
        \midrule
        狭 & \ipa{i}{i} & \ipa{u}{u} \\
        半狭 & \ipa{e}{e} & \ipa{o}{o} \\
        広 & \ipa{a}{a} & \\
        \bottomrule
    \end{tabular}
\end{table}

\begin{table}[H]
    \centering
    \caption{パラニーク・エルカでの子音}
    \label{table:consonants}
    \begin{tabular}{c|cccc}
        \toprule
             & 両唇音 & 舌頂音 & 舌背音 & 声門音 \\
        \midrule
        破裂音   & \ipa{p}{p} \ipa{v}{b} & \ipa{t}{t} \ipa{d}{d} & \ipa{k}{k} \ipa{g}{g} & \\
        鼻音     & \ipa{m}{m}            & \ipa{n}{n}            &                       & \\
        ふるえ音 &                       & \ipa{r}{r}            &                       & \\
        摩擦音   &                       & \ipa{s}{s} \ipa{z}{z} &                       & \ipa{h}{h} \\
        \bottomrule
    \end{tabular}
\end{table}

\begin{table}[H]
    \centering
    \caption{パラニーク・エルカでの子音(拗音)}
    \label{table:consonants-h}
    \begin{tabular}{ccccc}
        \toprule
        \ipa{th}{\t{tS}}     & \ipa{zh}{\t{dZ}}      & \ipa{rh}{r\super{j}} & \ipa{sh}{\c{c}}       & \ipa{f}{F} \\
        \ipa{ph}{p\super{j}} & \ipa{vh}{b\super{j}}  & \ipa{kh}{k\super{j}} & \ipa{gh}{g\super{j}}  &
        \ipa{mh}{m\super{j}} \\
        \ipa{nh}{\textltailn} &                      &                       & \\
        \bottomrule
    \end{tabular}
\end{table}

\section{促音・撥音}

子音を重ねると促音になる。例えば \emph{rakka} のとき発音は \textipa{/ra.k:a/} になる。
また、撥音に \ipa{n}{\textsyllabic{n}} が存在する。

\section{アクセント}

% TODO: TBA

\section{書記と発声の細則}

\subsection{語の接続による発音の区切り}

状態詞・供与詞の接続などにより子音と母音が一続きになる(例:\emph{sis-} + \emph{asteri})ときは \emph{'}で接続部を区切る(例:\emph{sis'asteri})。

\subsection{子音そのものの発音}

例えば子音 K や子音 M そのものを発音する(英語で言う「エー、ビー、シー……」)ときは大文字は接尾辞 \emph{-a}(例:「K」は \emph{Ka})、小文字は接尾辞 \emph{-ea} (例:「k」は \emph{kea})を子音につけて発音する。
