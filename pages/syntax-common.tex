\chapter{統語論(一般)}

\section{文の語順}

基本的な語順は \textbf{VSO} である。
供与詞による表現に置き換えると V\emph{k}a\emph{s} である。

\subsection{動詞のみ(V)}

\begin{itembox}[l]{例}
    \begin{pindent}
        \noindent
        Veiri.「起きる」

        \noindent
        Vanti.「行く」
    \end{pindent}
\end{itembox}

\subsection{動詞と主語(VS)}

\begin{itembox}[l]{例}
    \begin{pindent}
        \noindent
        Tantik parva.「塵が残る」

        \noindent
        Fekrik ne.「貴方が近づく」
    \end{pindent}
\end{itembox}

\subsection{動詞と主語と目的語(VSO)}

\begin{itembox}[l]{例}
    \begin{pindent}
        \noindent
        Parikas ni \emph{paranik erka}.「私は『パラニーク・エルカ』を話す。」

        \noindent
        Nasrikas ni varda. 「私は真実を聞く」
    \end{pindent}
\end{itembox}

\subsection{動詞と目的語(VO)}
供与詞 \emph{k} がない場合、何が主語になるかは文脈に依存する。

\begin{itembox}[l]{例}
    \begin{pindent}
        \noindent
        Esrikaris vestera.「物事を思い出せない」

        \noindent
        Sisnasis ferissa.「花を探しましょう」
    \end{pindent}
\end{itembox}

\section{疑問文}

疑問文はすべて平叙文の最初の語に状態詞 \emph{eus} を付加して表現する。
「はい」か「いいえ」で答えられる文はこの状態詞の付加だけでよい。
一方で疑問の対象がある場合は対象に応じた疑問代名詞を適当な供与詞で示す。

どちらの疑問文においても文末に「?」を置くことは必須ではない。

\section{繋辞 ai}

例えば「A は B だ」のように A と B の等価表現を実現するためには動詞 \emph{ai} を用いる。
A に対しては供与詞 \emph{k} を用い B に対しては供与詞 \emph{s} を用いる。

A または B が自明である場合(例えば Ais ni.とすれば「(それをしたのは・それは)私だ」となる)はそれぞれの供与詞とその項を省略しても構わない。

\begin{itembox}[l]{単純な「AはBである」の例}
    \begin{pindent}
        \noindent
        Aikas ne ritmike. \\
        「貴方は白い」
    \end{pindent}
\end{itembox}

\section{存在 xx}

% TODO: TBA