\chapter{統語論(一般)}

\section{文全体の語順}

基本的な語順は \textbf{VSO} である。ただし状態詞 \emph{-sy} の場所によっては VOS になりうる。
主語がない(=供与詞\emph{-sy} により付加されない)場合単純に VO のみとなる。

一般に文は動詞ないし名詞から始まるが、これを欠いた文も構築できる。
しかしその場合状態詞・供与詞の記述が特殊になる。

\subsection{動詞のみ(V)}

\begin{itembox}[l]{例}
    \begin{pindent}
        \noindent
        veriai.「起きる」

        \noindent
        zarai.「行く」
    \end{pindent}
\end{itembox}

\subsection{動詞と主語(VS)}

\begin{itembox}[l]{例}
    \begin{pindent}
        \noindent
        derai-sy deria.「人は死ぬ」(\emph{-sy} は主語を項に取る供与詞である)

        \noindent
        benerai-sy ry.「私は悔やむ」
    \end{pindent}
\end{itembox}

\subsection{動詞と主語と目的語(VSO)}

\begin{itembox}[l]{例}
    \begin{pindent}
        \noindent
        parai-sy-se ry \emph{raipara-re ry}. 「私は \emph{raipara-re ry} を話す」

        \noindent
        zerai-sy-se re ry. 「貴方は私を愛する」
    \end{pindent}
\end{itembox}

\subsection{動詞と目的語(VO)}
供与詞 \emph{-sy} がない場合、何が主語になるかは文脈に依存する。

\begin{itembox}[l]{例}
    \begin{pindent}
        \noindent
        ny-kyvenai-se roea.「迷いを知らない」

        \noindent
        rea-noadoai-se meria.「星を数えましょう」
    \end{pindent}
\end{itembox}

\subsection{核たる名詞・動詞がない文}
\label{section:sentence-without-noun-verb}

例えば「彼が」や「何を?」といった文を作るとする。
このとき文頭に動詞・名詞が存在しないながらに状態詞・供与詞は存在する宙ぶらりんな状態になる。
このとき該当する供与詞は\textbf{ハイフンなし}(\cref{table:list-of-punctuation} 参照)に記述する。
状態詞が存在するならば、最後の状態詞の直後に供与詞を\textbf{ハイフンつき}で記述する。

\begin{itembox}[l]{供与詞のみの場合}
    \begin{pindent}
        \noindent
        sy ryna! 「\textbf{我々が!}」

        \noindent
        sy-ne ry re.「私が君に」
    \end{pindent}
\end{itembox}

\begin{itembox}[l]{状態詞・供与詞が混在する場合}
    \begin{pindent}
        \noindent
        ay-hy reky? ay-sy ro-dezai?「君達なのか?壊したのは?」\\
        = ay-ai-sy-hy ro-dezai reky?
    \end{pindent}
\end{itembox}

\section{疑問文}

疑問文のうち、「はい」か「いいえ」で返答されうるものは通常の文の主たる動詞に状態詞 \emph{ay-} を付加して表現する。
一方で何がわからないか(または知りたいか)といった疑問の対象が明らかな場合は、状態詞 \emph{-ay} を付加した上で対象に対応する疑問詞を通常の供与詞を通じて表現する。

\begin{itemize}
    \item Seni: ay-enogai? 「食べますか?」 \\ Romi: reene. 「はい」(または)nene. 「いいえ」
    \item Kyrani: ay-ri-honorai-se kandea? 「何を歌いましょうか?」\\ Renei: ri-ai-se «reia-re reda». 「『夜の月』を歌いましょう」
\end{itemize}

なお、文末に「?」をつけることは必須ではないが、つけてもよい。