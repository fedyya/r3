\chapter{統語論(一般)}

\section{文全体の語順}

基本的な語順は \textbf{VSO} である。ただし mod \emph{-sy} の場所によっては VOS になりうる。
主語がない(= sup \emph{-sy} により付加されない)場合単純に VO のみとなる。

\subsection{動詞のみ(V)}

\begin{itemize}
    \item veriai.   「起きる」
    \item zarai.    「行く」
    \item ameriai.  「泣く」
\end{itemize}

\subsection{動詞と主語(VS)}
\begin{itemize}
    \item derai-sy deria.   「人は死ぬ」(\emph{-sy} は主語を項に取る sup である)
    \item benerai-sy ry.    「私は悔やむ」
    \item akyai-sy re.      「貴方は捧げる」
\end{itemize}

\subsection{動詞と主語と目的語(VSO)}

\begin{itemize}
    \item parai-sy-se ry \emph{raipara-re ry}. 「私は r3\footnote{raipara-re ry の略} を話す」
    \item zerai-sy-se re ry. 「貴方は私を愛する」
    \item dea-dai-se-ne minura ry. 「秘密を私に与えよ(=教えよ)」
\end{itemize}

\subsection{動詞と目的語(VO)}
主語がない場合、何が主語になるかは文脈に依存する。

\begin{itemize}
    \item rea-noadoai-se meria. 「星を数えましょう」
    \item ny-kyvenai-se roea. 「迷いを知らない」
\end{itemize}

\clearpage

\section{疑問文}

疑問文のうち、「はい」か「いいえ」で返答されうるものは通常の文の主たる動詞に mod \emph{ay-} を付加して表現する。
一方で何がわからないか(または知りたいか)といった疑問の対象が明らかな場合は、mod \emph{-ay} を付加した上で対象に対応する疑問詞を通常の sup を通じて表現する。

\begin{itemize}
    \item Seni: ay-enogai? 「食べますか?」 \\ Romi: reene. 「はい」(または)nene. 「いいえ」
    \item Kyrani: ay-ri-honorai-se kandea\footnote{kandea は名詞に対応する疑問詞(=何)である。}? 「何を歌いましょうか?」\\ Renei: ri-ai-se «reia-re reda». 「『夜の月』を歌いましょう」
\end{itemize}

なお、文末に「?」をつけることは必須ではないが、つけてもよい。