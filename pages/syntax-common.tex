\chapter{統語論(一般)}

\section{文の語順}

基本的な語順は \textbf{VSO} である。
供与詞による表現に置き換えると V\emph{k}a\emph{s} である。

\subsection{動詞のみ(V)}

\begin{itembox}[l]{例}
    \begin{pindent}
        \noindent
        Veiri.「起きる」

        \noindent
        Vanti.「行く」
    \end{pindent}
\end{itembox}

\subsection{動詞と主語(VS)}

\begin{itembox}[l]{例}
    \begin{pindent}
        \noindent
        Tantik parva.「塵が残る」

        \noindent
        Fekrik ne.「貴方が近づく」
    \end{pindent}
\end{itembox}

\subsection{動詞と主語と目的語(VSO)}

\begin{itembox}[l]{例}
    \begin{pindent}
        \noindent
        Parikas ni \emph{paranik erka}.「私は『パラニーク・エルカ』を話す。」

        \noindent
        Nasrikas ni varda. 「私は真実を聞く」
    \end{pindent}
\end{itembox}

\subsection{動詞と目的語(VO)}
供与詞 \emph{k} がない場合、何が主語になるかは文脈に依存する。

\begin{itembox}[l]{例}
    \begin{pindent}
        \noindent
        Esrikaris vestera.「物事を思い出せない」

        \noindent
        Sisnasis ferissa.「花を探しましょう」
    \end{pindent}
\end{itembox}

\section{否定文 es-}

否定文は平叙文の適当な語に状態詞 \emph{es} を修飾して表現する。
もっとも \emph{es} は単語の極性を反転するのに使われることもある。

\begin{itembox}[l]{否定文の例}
    \begin{pindent}
        \noindent
        \emph{Es}rikarizak eszisrikari ni. \\
        「(私は)思い出せないし、思い出したくもない。」
    \end{pindent}
\end{itembox}

\section{疑問文 aus-}

疑問文は平叙文の文頭に状態詞 \emph{aus} を修飾して表現する。

「はい」か「いいえ」で答えられる文はこの状態詞を修飾するだけでよい。
一方で疑問の対象がある場合は対象に応じた疑問代名詞を適当な供与詞で示す。

どちらの疑問文においても文末に「?」を置くことは必須ではない。

\begin{itembox}[l]{疑問文の例}
    \begin{pindent}
        \noindent
        \emph{Aus}rikarik ne? \\
        「(貴方は)思い出せる?」
    \end{pindent}
\end{itembox}

\section{命令文 zis・sis}

命令文は文頭に状態詞 \emph{zis} ないし \emph{sis} を修飾して表現する。

\emph{zis}と\emph{sis}の違いはその要求の強さの程度にある。\emph{zis}は強い表現であり、例えば上司が部下に命令するといった状況が考えられる。\emph{sis}は弱い表現であり、例えば友人間の緩い誘いや要求に用いられる。

\begin{itembox}[l]{疑問文(強勢)の例}
    \begin{pindent}
        \noindent
        \emph{Zis}navirinik mirke! \\
        「三回繰り返せ!」
    \end{pindent}
\end{itembox}

\begin{itembox}[l]{疑問文(弱勢)の例}
    \begin{pindent}
        \noindent
        \emph{Sis}nestrikas ne zaranik vismike! \\
        「(貴方は)赤いお花を集めてきて!」
    \end{pindent}
\end{itembox}

\section{繋辞 asi}

例えば「A は B だ」のような A と B の等価表現を実現するには動詞 \emph{asi} を用いる。
A に対しては供与詞 \emph{k} を用い B に対しては供与詞 \emph{s} を用いる。

A または B が自明である場合(例えば Ais ni.とすれば「(それをしたのは・それは)私だ」となる)はそれぞれの供与詞とその項を省略しても構わない。

\begin{itembox}[l]{単純な「AはBである」の例}
    \begin{pindent}
        \noindent
        Asikas ne ritmike. \\
        「貴方は白い」
    \end{pindent}
\end{itembox}

\section{存在 ari}

例えば「Aがいる(存在する)」のような存在表現を実現するには動詞 \emph{ari} を用いる。
A に対しては供与詞 \emph{k} を用いる。

\begin{itembox}[l]{単純な「Aがある」の例}
    \begin{pindent}
        \noindent
        Arik ninik nasarissa. \\
        「調査員の私がいる」
    \end{pindent}
\end{itembox}
