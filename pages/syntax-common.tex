\chapter{統語論(一般)}

\section{文全体の語順}

基本的な語順は \textbf{VSO} である。ただし mod \emph{-sy} の場所によっては VOS になりうる。
主語がない(= sup \emph{-sy} により付加されない)場合単純に VO のみとなる。

一般に文は動詞ないし名詞から始まるが、これを欠いた文も構築できる。
しかしその場合 mod・sup の記述が特殊になる。

\subsection{動詞のみ(V)}

\begin{itembox}[l]{例}
    \begin{pindent}
        \noindent
        veriai.「起きる」

        \noindent
        zarai.「行く」
    \end{pindent}
\end{itembox}

\subsection{動詞と主語(VS)}

\begin{itembox}[l]{例}
    \begin{pindent}
        \noindent
        derai-sy deria.「人は死ぬ」(\emph{-sy} は主語を項に取る sup である)

        \noindent
        benerai-sy ry.「私は悔やむ」
    \end{pindent}
\end{itembox}

\subsection{動詞と主語と目的語(VSO)}

\begin{itembox}[l]{例}
    \begin{pindent}
        \noindent
        parai-sy-se ry \emph{raipara-re ry}. 「私は \emph{raipara-re ry} を話す」

        \noindent
        zerai-sy-se re ry. 「貴方は私を愛する」
    \end{pindent}
\end{itembox}

\subsection{動詞と目的語(VO)}
sup \emph{-sy} がない場合、何が主語になるかは文脈に依存する。

\begin{itembox}[l]{例}
    \begin{pindent}
        \noindent
        ny-kyvenai-se roea.「迷いを知らない」

        \noindent
        rea-noadoai-se meria.「星を数えましょう」
    \end{pindent}
\end{itembox}

\subsection{核たる名詞・動詞がない文}
\label{section:sentence-without-noun-verb}

例えば「彼が」や「何を?」といった文を作るとする。このとき文頭に動詞・名詞が存在しないながらに mod・sup は存在する宙ぶらりんな状態になる。このとき該当する sup は\textbf{ハイフンなし}(\cref{table:list-of-punctuation} 参照)に記述する。mod が存在するならば、最後の mod の直後に sup を\textbf{ハイフンつき}で記述する。

\begin{itembox}[l]{sup のみの場合}
    \begin{pindent}
        \noindent
        sy ryna! 「\textbf{我々が!}」

        \noindent
        sy-ne ry re.「私が君に」
    \end{pindent}
\end{itembox}

\begin{itembox}[l]{mod・sup が混在する場合}
    \begin{pindent}
        \noindent
        ay-hy reky? ay-sy ro-dezai?「君達なのか?壊したのは?」\\
        = ay-ai-sy-hy ro-dezai reky?
    \end{pindent}
\end{itembox}

\section{疑問文}

疑問文のうち、「はい」か「いいえ」で返答されうるものは通常の文の主たる動詞に mod \emph{ay-} を付加して表現する。
一方で何がわからないか(または知りたいか)といった疑問の対象が明らかな場合は、mod \emph{-ay} を付加した上で対象に対応する疑問詞を通常の sup を通じて表現する。

\begin{itemize}
    \item Seni: ay-enogai? 「食べますか?」 \\ Romi: reene. 「はい」(または)nene. 「いいえ」
    \item Kyrani: ay-ri-honorai-se kandea? 「何を歌いましょうか?」\\ Renei: ri-ai-se «reia-re reda». 「『夜の月』を歌いましょう」
\end{itemize}

なお、文末に「?」をつけることは必須ではないが、つけてもよい。