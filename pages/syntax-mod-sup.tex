\chapter{統語論(mod・sup)}

\section{supの項の解決}

\subsection{sup の役割・基礎的な読み解き方}

Supplier、略して sup は動詞または名詞と続く語の関係性を示す重要な概念である。いわゆる格(例:~が・~を・~に、など)など各種の文法事項の大半がこの sup で解決される。sup は基本的に修飾する名詞・動詞に続く語を sup の並びに応じて\textbf{左から順に}項に取っていく。

sup は常に一つだけ語、名詞句、動詞句などを項に取る。逆説的に複数の語(の塊)が一つの sup で解決されることはない。言い換えるなら上述の通り左から順に項(語)を取る時、一つだけ取ることはあっても二つ連続して取ることはない。

以下に単純な例を挙げる:

\begin{itemize}
    \item zeai-sy ry. 「私は愛する」\\ sup \emph{-sy}(~は) が ry(名詞:私) を項に取っている。
    \item zeai-sy-se ry re. 「私は貴方を愛する」\\ sup \emph{-sy} が ry、sup \emph{-se}(~を)が re(名詞:貴方)を項に取っている。
\end{itemize}

\subsection{入れ子になった sup の解決}

例えば「zeai-sy-se ry re-re perie.」という文章を考える。意味は「私は可愛い貴方を愛する」である。ここで注目するべきは、sup \emph{-re} が項に取っている \emph{re} が sup \emph{-re} を持っていることである。

このような場合も常に左から項を解決する。即ち以下のように解決を進める:

\begin{enumerate}
    \item sup \emph{-sy} が最初の項 ry を取る。\\ →「私(ry)は(-sy)愛する(zeai)」
    \item sup \emph{-se} が次の項 re を取る。 \\ →「貴方(re)を(-se)」
    \item re-re の sup \emph{-re} が次の項 perie を取る。 \\ →「可愛い(perie)の(-re)貴方(re)」=「可愛い貴方」\\ ここで re-re perie は一塊の名詞句として扱われる。
    \item それぞれの項をまとめて sup の解決を終了する。
\end{enumerate}

\section{mod の解決}

sup と違い Modifier、略して mod は一切の項を持たないので、sup より解決は極めて簡単である。mod は常に対象の動詞・名詞の前につく。対象が直ちに定まるため、mod の意味を直接対象に与えるだけで良い。以下は例である。

\begin{itemize}
    \item ri-honorai.「歌いましょう」\\ mod \emph{ri-} は相対的に弱い呼びかけ・誘いの意味合いを持つ。
    \item ho-derai-sy ria. 「人は死ぬ」\\ mod \emph{ho-} は将来起こることを示す。
\end{itemize}

\section{sup の修飾品詞による意味の変更}

一部の sup はそれが修飾する語の品詞(動詞か名詞)によって意味を変える。これを \textbf{sup のポリモーフィズム} と呼ぶ。具体的には sup \emph{-re} が該当する。

\subsection{-re のポリモーフィズム}

便宜的に文を「A-re B」とする。

\begin{itemize}
    \item A=名詞・B=形容詞 \\ → B は形容詞のままである。
    \item A=動詞・B=形容詞 \\ → B が副詞化する。
\end{itemize}