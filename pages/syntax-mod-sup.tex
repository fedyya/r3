\chapter{統語論(状態詞・供与詞)}

\section{供与詞の解決}

供与詞は語と語の関係を示す。格(例:~が・~を・~に、など)・文の接続は供与詞で処理する。
供与詞は後に続く項(動詞句・名詞句を含む)を供与詞の並びに応じて、\textbf{右から順に、常に一つだけ}取る。

動詞句・名詞句は複数語から構成されうるが、供与詞は句を一つの塊としか見ず語単位で解釈することはしない。
例えば「美しい花」という名詞句があるとき供与詞は「美しい花」という塊として見る。
これを「美しい」と「花」と分解して「美しい」だけを項に取ることはない。

\section{品詞による意味の変化}

一部の供与詞、特に \emph{nik} は接受語の品詞(動詞か名詞)によって項の意味を変える。

例えば、項が形容詞として接受語が動詞だと、形容詞である項を副詞として処理する。
一方で接受語が名詞であれば項は形容詞のままで扱う。

整理すると以下のようになる。

\begin{itemize}
    \item A=名詞・B=形容詞 \\ → B は形容詞のままである。
    \item A=動詞・B=形容詞 \\ → B が副詞化する。
\end{itemize}

\section{状態詞の解決}

状態詞はその語の状態を示す。文法事項のうち、数・性・相・態・時制は状態詞によって示される。

状態詞は一切の項を持たないため、解決は供与詞より簡単である。
状態詞は常に修飾している語と一対一で対応する。
よって意味を考えるときは状態詞の意味をその語に足すだけでよい。