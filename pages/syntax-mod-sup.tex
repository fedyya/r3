\chapter{統語論(状態詞・供与詞)}

\section{供与詞の項の解決}

\subsection{供与詞の役割・基礎的な読み解き方}

供与詞は動詞または名詞と続く語の関係性を示す重要な概念である。
いわゆる格(例:~が・~を・~に、など)など各種の文法事項の大半がこの供与詞で解決される。
供与詞は基本的に修飾する名詞・動詞に続く語を供与詞の並びに応じて\textbf{左から順に}項に取っていく。

供与詞は常に一つだけ語、名詞句、動詞句などを項に取る。
逆説的に複数の語(の塊)が一つの供与詞で解決されることはない。
言い換えるなら上述の通り左から順に項(語)を取る時、一つだけ取ることはあっても二つ連続して取ることはない。

以下に単純な例を挙げる:

\begin{itemize}
    \item zeai-sy ry. 「私は愛する」\\ 供与詞 \emph{-sy}(~は) が ry(名詞:私) を項に取っている。
    \item zeai-sy-se ry re. 「私は貴方を愛する」\\ 供与詞 \emph{-sy} が ry、供与詞 \emph{-se}(~を)が re(名詞:貴方)を項に取っている。
\end{itemize}

\subsection{入れ子になった供与詞の解決}

例えば「zeai-sy-se ry re-re perie.」という文章を考える。意味は「私は可愛い貴方を愛する」である。ここで注目するべきは、供与詞 \emph{-re} が項に取っている \emph{re} が供与詞 \emph{-re} を持っていることである。

このような場合も常に左から項を解決する。即ち以下のように解決を進める:

\begin{enumerate}
    \item 供与詞 \emph{-sy} が最初の項 ry を取る。\\ →「私(ry)は(-sy)愛する(zeai)」
    \item 供与詞 \emph{-se} が次の項 re を取る。 \\ →「貴方(re)を(-se)」
    \item re-re の供与詞 \emph{-re} が次の項 perie を取る。 \\ →「可愛い(perie)の(-re)貴方(re)」=「可愛い貴方」\\ ここで re-re perie は一塊の名詞句として扱われる。
    \item それぞれの項をまとめて供与詞の解決を終了する。
\end{enumerate}

\section{状態詞の解決}

状態詞は一切の項を持たない。よって解決は供与詞より極めて簡単である。

状態詞は常に対象の動詞・名詞の接頭辞として現れる。状態詞は常に対象の語と一対一で対応する。
これにより対象が直ちに定まる。よって解決は状態詞の意味を対象に与えるだけでよい。

以下は例である。

\begin{itemize}
    \item ri-honorai.「歌いましょう」\\ 状態詞 \emph{ri-} は相対的に弱い呼びかけ・誘いの意味合いを持つ。
    \item ho-derai-sy ria. 「人は死ぬ」\\ 状態詞 \emph{ho-} は将来起こることを示す。
\end{itemize}

\section{供与詞の修飾品詞による意味の変更}

一部の供与詞はそれが修飾する語の品詞(動詞か名詞)によって意味を変える。これを \textbf{sup のポリモーフィズム} と呼ぶ。具体的には供与詞\emph{-re} が該当する。

\subsection{-re のポリモーフィズム}

便宜的に文を「A-re B」とする。

\begin{itemize}
    \item A=名詞・B=形容詞 \\ → B は形容詞のままである。
    \item A=動詞・B=形容詞 \\ → B が副詞化する。
\end{itemize}