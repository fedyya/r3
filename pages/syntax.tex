\section{統語論}

\subsection{文全体の語順}

基本的な語順は VSO である。ただし Modifier \emph{-sy} の場所によっては VOS になりうる。
主語がない(\emph{-sy} により付加されない)場合単純に VO のみとなる。

\subsubsection{動詞のみ(V)}

\begin{itemize}
    \item veriai.   「起きる」
    \item zarai.    「行く」
    \item ameriai.  「泣く」
\end{itemize}

\subsubsection{動詞と主語(VS)}
\begin{itemize}
    \item derai-sy deria.   「人は死ぬ」(\emph{-sy} は主語を項に取る Supplier である)
    \item benerai-sy ry.    「私は悔やむ」
    \item akyai-sy re.      「貴方は捧げる」
\end{itemize}

\subsubsection{動詞と主語と目的語(VSO)}

\subsubsection{動詞と目的語(VO)}

\subsection{Modifier・Supplier の語順}